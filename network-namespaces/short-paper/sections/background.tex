\section{Background Concepts} \label{sec:background}

This project builds on two foundational concepts of computer systems: containers, and checkpoint-restore.

\subsection{\criu: Checkpoint-Restore in Userspace}

Checkpoint/Restore in Userspace (\criu) is an open-source C/R tool~\cite{criu-main-page}.
Introduced in 2011, it's distinctive feature is that it is mainly implemented in user space, rather than in the kernel, by using existing interfaces~\cite{Reber2016}.
One of the most important interface is \texttt{ptrace}~\cite{ptrace-manpage}, as it relies on it for seizing the target process.
For other interfaces, several patches have been pushed to the mainline kernel by \criu developers~\cite{criu-kernel-patches}.
The project is currently under active development~\cite{criu-github}, and its main focus is to support the migration of containers.

The checkpointing process starts with the process identifier (PID) of a process group leader provided by the user through the command line using the \texttt{--tree} option~\cite{criu-checkpoint}.

Once all tasks are frozen, \criu collects all the information it can about the task's resources.
File descriptors and registers are dumped through a \texttt{ptrace} interface and are parsed from \texttt{/proc/\$pid/fd} and \texttt{/proc/\$pid/stat} respectively.

During the restore process, \criu morphs into the to-be-restored task.
Since we checkpoint process trees rather than single processes, \criu must \texttt{fork} itself several times to recreate the original PID tree.
Then \criu restores all basic task resources such as file descriptors and namespaces, and resumes the process.

\subsection{\runc: OCI-compliant container runtime}

Originally developed at Docker, \runc is a lightweight container runtime aimed to provide low-level interaction with containers.
In 2015~\cite{introducing-runc}, Docker open-sourced the component and transferred ownership to the Open Container Initiative (OCI), who has since then lead the project in a fashion similar to that of the Linux Foundation.
Since then, several container engines such as \textsc{Podman} (\url{https://podman.io/}) and \textsc{CRI-O} (\url{https://cri-o.io/}) have made \runc their runtime of choice.

The OCI has since then released specifications for container runtimes, engines, images, and image distribution.
\runc is nowadays an OCI-compliant container runtime (it is, in fact, the reference implementation).

Users are encouraged to interact with containers through container engines, but \runc itself provides an interface to create, run, and manage containers natively.
Integration with \criu has to be done on a per-project basis, and \runc has the most advanced and stable integration.
Therefore, we decided to use it to manage our containers.

Running a container with \runc is slightly different than doing it in, let's say, Docker, as the user's interaction with the underlying system is more direct.
In particular, in \runc there is no notion of \textit{images}.
To run a container, a user must provide a specification file (\texttt{config.json}) and a root file system in a directory (\texttt{rootfs}).
Through the specification file several low-level options such as namespaces, control groups, and capabilities can be configured.
