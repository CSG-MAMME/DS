\begin{abstract}
    One of the major hurdles for containers to become the cloud tenants' choice for application sandboxing is fault-tolerancy and load balancing.
    Checkpoint/Restore provides an efficient solution as it enables to save snapshots of running programs from which execution can be restored.
    Checkpointing and live migration for virtual machines has been around for several years now and, albeit less efficient, it provides with the robustness that a datacenter requires.
    However, checkpoint/restore (C/R) of running containers is still an open and active area of research.
    The biggest attempt to filling such a gap is Checkpoint-Restore in Userspace (CRIU) a software tool designed to dump, manage, and restore running processes by leveraging existing interfaces of the Linux Kernel.
    It does so completely from user-space and fully transparently.

    In this work we present a proof of concept implementation for live migration of established TCP connections of runC containers.
    Although C/R of established sockets is already available with \criu, migrating such a connection is left to the user, hence the novelty of our holistic approach.
    We also present a benchmark of the downtime experienced by a client when a network-intensive server is dumped and restored.
    Lastly, this short paper is a work in progress towards achieving C/R of container clusters to provide tools such as \textsc{Kubernetes} with fast restore times in the event of a node failure.
\end{abstract}
